% Compile with: xelatex main && xelatex main  (or use latexmk)
\documentclass[10pt,aspectratio=169]{beamer}

% ---- Theme & look ----
\usetheme[numbering=fraction,progressbar=frametitle]{metropolis} % modern clean theme
\metroset{block=fill, titleformat frame=smallcaps, sectionpage=progressbar, numbering=fraction}

% ---- Language & fonts (Chinese + Latin, use XeLaTeX) ----
\usepackage[UTF8]{ctex}            % Chinese support
\usepackage{fontspec}

% ---- Common packages ----
\usepackage{amsmath,amssymb,mathtools}
\usepackage{physics}
\usepackage{siunitx}
\usepackage{graphicx}
\usepackage{booktabs}
\usepackage{hyperref}
\usepackage{tabularx}
\usepackage{tikz}
\usetikzlibrary{arrows.meta,positioning}
\usepackage[ruled,vlined]{algorithm2e}
\usepackage{listings}
\lstset{basicstyle=\ttfamily\footnotesize,breaklines=true,frame=single}

% ---- Colors (brand-ish) ----
\definecolor{XBlue}{HTML}{2E77BB}
\definecolor{XGold}{HTML}{C7A008}
% 新增:确保整体是白底黑字,避免整页发灰
\setbeamercolor{background canvas}{bg=white}
% 恢复主题细节配色


\setbeamercolor{title}{fg=black}
\setbeamercolor{block body}{bg=XBlue!5}

% ---- Meta ----
\title[短标题]{Raytracing with Radiocity and Galerkin Method}
\author[Your Name]{Yuguang Yao\\\small École Polytechnique}


% Optional: section table of contents at each section start
% 修复:限制 dark 背景只用于该帧,之后恢复 light


% ---- Document ----
\begin{document}

% Title page
\begin{frame}[plain]
  \titlepage
\end{frame}


\begin{frame}{问题背景}
    母方程:
$$
\underbrace{L_o(x,\omega_o)}_{\text{出射辐亮度}}
=
\underbrace{L_e(x,\omega_o)}_{\text{自发光}}
+\;
\int_{\Omega^+}
\underbrace{f_r(x,\omega_i,\omega_o)}_{\text{BRDF}}
\;
\underbrace{L_i(x,\omega_i)}_{\text{入射辐亮度}}
\;
\underbrace{(\omega_i\!\cdot\!n)}_{\text{余弦项}}
\;d\omega_i
$$

\begin{itemize}
    \item $L$:辐亮度 Radiance,单位 $W\,m^{-2}\,sr^{-1}$(“每面积每立体角的光功率密度”)。
    \item $f_r$:BRDF,单位 $sr^{-1}$;守恒要求 $\int_{\Omega^+} f_r(\omega_i,\omega_o)(\omega_i\!\cdot\!n)\,d\omega_i \le 1$。
    \item $(\omega_i\!\cdot\!n)$:余弦项,体现“斜着来的光有效面积更小”。
    \item $\Omega^+$:法线半球。
\end{itemize}
\end{frame}



\begin{frame}{Radiosity}
    积分方程:
$$
L_o(x,\omega_o)=L_e(x,\omega_o)
+\int_{A} f_r(x,\omega_i,\omega_o)\,L_o(y,-\omega_i)\,G(x,y)\,V(x,y)\,dA_y
$$

\begin{itemize}
    \item $V(x,y)\in\{0,1\}$:可见性。
    \item $G(x,y)=\dfrac{(\omega_i\!\cdot\!n_x)(-\omega_i\!\cdot\!n_y)}{\|x-y\|^2}$:距离衰减+余弦。
    \item $\omega_i=\dfrac{y-x}{\|y-x\|}$:从 $x$ 指向 $y$ 的方向。
    \item $f_r(x,\omega_i,\omega_o)=\frac{\rho}{\pi}$:漫反射.
\end{itemize}
\end{frame}

\begin{frame}{Galerkin Method}
   Galerkin 方法:
  $$
  \quad u(x)-\lambda\!\int K(x,t)u(t)\,dt=f(x).
  $$

  选一组基函数 $\{\phi_i\}_{i=1}^n$,令未知近似

  $$
  u_n(x)=\sum_i c_i\,\phi_i(x).
  $$

  要求残差 $r=u_n-\lambda K[u_n]-f$ 对每个基函数正交:

  $$
  \langle r,\phi_j\rangle = 0\ \Rightarrow\ 
  \sum_i \big(\langle \phi_i,\phi_j\rangle -\lambda\langle K[\phi_i],\phi_j\rangle\big)c_i = \langle f,\phi_j\rangle.
  $$
  得到一个全局线性方程组 $A c = b$.

\end{frame}

\begin{frame}{Radiocity + Galerkin}
    Radiocity + Galerkin:
    P0基:每个面片 $A_i$ 上 $\phi_i(x)=\mathbf{1}_{A_i}(x)$。
    \begin{itemize}
        \item 方程:$B = E + \rho K[B]$,核 $K(x,y)=G(x,y)V(x,y)$。
        \item 取基 $\{\phi_i\}$,Galerkin:$\langle \phi_j, B\rangle = \langle \phi_j, E\rangle + \rho \langle \phi_j, K[B]\rangle$。
        \item 写成矩阵:$(M - \rho A) c = b$,其中
        \begin{itemize}
            \item $M_{ji}=\langle \phi_j,\phi_i\rangle$(P0 时 $=A_j\delta_{ij}$);
            \item $A_{ji}=\int\!\!\int \phi_j(x)K(x,y)\phi_i(y)\,dA_x dA_y$;
            \item $b_j=\langle \phi_j,E\rangle$。
        \end{itemize}
    \end{itemize}
  归一化后就是 $(I-\rho F)B=E$。

\end{frame}

\begin{frame}{feasibility}
patches 数量在$N\sim 10^3$级别时,效果?

scene: 6个面的房间($A\sim 5$),1个光源面(1*1),一个五个面立方体(a=1)

patch density: $A_i\sim \rho$

$10*A^2\rho \sim 10^3\rightarrow \rho \sim 4$

因此每个面片的长度大概在 $0.25m$ 级别.
\end{frame}

\begin{frame}{可视化}
渲染时每个像素:
$$
\text{Pixel} \propto \int_{\lambda}\!\!\int_{t}\!\!\int_{\mathcal{A}_{\text{aperture}}}\!\!\int_{\mathcal{P}_{\text{pixel}}}
L_o\!\big(x(\xi),\omega(\xi),\lambda,t\big)\;
W_{\text{pix}}(\xi_{\text{pixel}})\, d\xi_{\text{pixel}}\, d\xi_{\text{aperture}}\, dt\, S(\lambda)\, d\lambda .
$$


\begin{itemize}
    \item $W_{\text{pix}}$:像素重建滤波(盒/三角/Tent、Mitchell、Lanczos)。
    \item $S(\lambda)$:光谱到感光元的敏感度(或 XYZ CMF / 显示原色敏感度)。
\end{itemize}
\end{frame}

\begin{frame}{可视化}
    \begin{figure}
        \centering
        \includegraphics[height=0.8\textheight]{outputs/Left_up_5560.png}
        \caption{$N_{Patch}=3129$}
    \end{figure}
\end{frame}



% ---- End ----
\end{document}
